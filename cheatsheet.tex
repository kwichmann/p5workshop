%%%%%%%%%%%%%%%%%%%%%%%%%%%%%%%%%%%%%%%%%
% Cheatsheet
% LaTeX Template
% Version 1.0 (12/12/15)
%
% This template has been downloaded from:
% http://www.LaTeXTemplates.com
%
% Original author:
% Michael Müller (https://github.com/cmichi/latex-template-collection) with
% extensive modifications by Vel (vel@LaTeXTemplates.com)
%
% License:
% The MIT License (see included LICENSE file)
%
%%%%%%%%%%%%%%%%%%%%%%%%%%%%%%%%%%%%%%%%%

%----------------------------------------------------------------------------------------
%	PACKAGES AND OTHER DOCUMENT CONFIGURATIONS
%----------------------------------------------------------------------------------------

\documentclass[11pt]{scrartcl} % 11pt font size

\usepackage[utf8]{inputenc} % Required for inputting international characters
\usepackage[T1]{fontenc} % Output font encoding for international characters

\usepackage[margin=0pt, landscape]{geometry} % Page margins and orientation

\usepackage{graphicx} % Required for including images

\usepackage{color} % Required for color customization
\definecolor{mygray}{gray}{.75} % Custom color

\usepackage{url} % Required for the \url command to easily display URLs

\usepackage[ % This block contains information used to annotate the PDF
colorlinks=false, 
pdftitle={p5 heatsheet}, 
pdfauthor={Kristian Wichmann}, 
pdfsubject={Compilation of functions and links}, 
pdfkeywords={p5, p5js, Cheatsheet}
]{hyperref}

\setlength{\unitlength}{1mm} % Set the length that numerical units are measured in
\setlength{\parindent}{0pt} % Stop paragraph indentation

\renewcommand{\dots}{\ \dotfill{}\ } % Fills in the right amount of dots

\newcommand{\command}[2]{#1~\dotfill{}~#2\\} % Custom command for adding a shorcut

\newcommand{\sectiontitle}[1]{\paragraph{#1} \ \\} % Custom command for subsection titles

%----------------------------------------------------------------------------------------

\begin{document}

\begin{picture}(297,210) % Create a container for the page content

%----------------------------------------------------------------------------------------
%	TITLE SECTION 
%----------------------------------------------------------------------------------------

\put(10,200){ % Position on the page to put the title
\begin{minipage}[t]{210mm} % The size and alignment of the title
\section*{Processing Community Day Copenhagen 2020 - p5 workshop - cheatsheet} % Title
\end{minipage}
}

%----------------------------------------------------------------------------------------
%	FIRST COLUMN SPECIFICATION
%----------------------------------------------------------------------------------------

\put(10,180){ % Divide the page
\begin{minipage}[t]{85mm} % Create a box to house text

%----------------------------------------------------------------------------------------
%	GITHUB LINK
%----------------------------------------------------------------------------------------

\sectiontitle{Workshop github page}
			
\url{github.com/kwichmann/p5workshop}\\

%----------------------------------------------------------------------------------------
%	BASIC GRAPHICS
%----------------------------------------------------------------------------------------				
			
\sectiontitle{Basic graphics}

\textbf{Canvas} \\
\command{\texttt{createCanvas(width, height)}}{Create}
\command{\texttt{background(r, g, b)}}{Background color}

\textbf{Points and lines}\\
\command{\texttt{point(x, y)}}{Point at $(x,y)$}
\command{\texttt{line(x1, y1, x2, y2)}}{From $(x_1,y_1)$ to $(x_2,y_2)$}

\textbf{Style of points, lines, and outlines}\\
\command{\texttt{strokeWeight(size)}}{Size/width}
\command{\texttt{stroke(r, g, b)}}{Color}
\command{\texttt{noStroke()}}{Disable}

\textbf{Geometric shapes}\\
\command{\texttt{rect(x, y, width, height)}}{Rectangle at $(x,y)$}
\command{\texttt{ellipse(x, y, radius)}}{Circle with center$(x,y)$}
\command{\texttt{ellipse(x, y, width, height)}}{Ellipse}
\command{\texttt{triangle(x1, y1, x2, y2, x3, y3)}}{Triangle}

\textbf{Style of fill}\\
\command{\texttt{fill(r, g, b)}}{Color}
\command{\texttt{noFill()}}{Disable}

%----------------------------------------------------------------------------------------
%	COLOR
%----------------------------------------------------------------------------------------	
\sectiontitle{Color arguments}

\command{$\cdots$\texttt{(r, g, b)}}{Red, green, and blue, 0-255}
\command{$\cdots$\texttt{(brightness)}}{Grayscale: 0 black, 255 white}

\textbf{Alpha:} 0 fully transparent, 255 fully opaque
\command{$\cdots$\texttt{(r, g, b, a)}}{Red, green, blue with alpha}
\command{$\cdots$\texttt{(brightness, a)}}{Grayscale with alpha}

%----------------------------------------------------------------------------------------

\end{minipage} % End the first column of text
} % End the first division of the page

%----------------------------------------------------------------------------------------
%	SECOND COLUMN SPECIFICATION 
%----------------------------------------------------------------------------------------

\put(105,180){ % Divide the page
\begin{minipage}[t]{85mm} % Create a box to house text

%----------------------------------------------------------------------------------------
%	COMMENTS
%----------------------------------------------------------------------------------------

\sectiontitle{Comments}

\command{\texttt{// Comment}}{Note to self. Is not executed}

%----------------------------------------------------------------------------------------
%	BUILT-IN VARIABLES
%----------------------------------------------------------------------------------------

\sectiontitle{Built-in variables}

\command{\texttt{width} and \texttt{height}}{Canvas size}
\command{\texttt{mouseX} and \texttt{mouseY}}{Current mouse position}
\command{\texttt{pmouseX} and \texttt{pmouseY}}{Previous mouse position}
\command{\texttt{key}}{Last key pressed}

%----------------------------------------------------------------------------------------
%	EVENTS
%----------------------------------------------------------------------------------------

\sectiontitle{Events}

\command{\texttt{function mousePressed() \{...\}}}{Mouse click}
\command{\texttt{function keyPressed() \{...\}}}{Keyboard input}
\command{\texttt{function preload() \{...\}}}{Done before setup}
					
%----------------------------------------------------------------------------------------
%	VARIABLES
%----------------------------------------------------------------------------------------				
					
\sectiontitle{Variables and constants}

\command{\texttt{let a}}{Declare variable $a$}
\command{\texttt{let b = 5}}{Declare and assign variable $b$}
\command{\texttt{const c = 42}}{Assign constant $c$}
\command{\texttt{d = 3 * a + d}}{Reassign value to variable $d$}
\command{\texttt{n += 3}}{Add 3 to variable $n$}
\command{\texttt{m -= 5}}{Subtract 4 from variable $m$}
\command{\texttt{x *= 2}}{Multiply variable $x$ by 2}
\command{\texttt{y /= 7}}{Divide variable $y$ by 7}
\command{\texttt{i++}}{Add 1 to variable $i$}
\command{\texttt{j++}}{Subtract 1 from variable $j$}

%----------------------------------------------------------------------------------------
%	CONSOLE
%----------------------------------------------------------------------------------------

\sectiontitle{The console}

\command{\texttt{console.log(x)}}{Write the value of $x$ in console}


%----------------------------------------------------------------------------------------

\end{minipage} % End the second column of text
} % End the second division of the page

%----------------------------------------------------------------------------------------
%	THIRD COLUMN SPECIFICATION 
%----------------------------------------------------------------------------------------

\put(200,180){ % Divide the page
\begin{minipage}[t]{85mm} % Create a box to house tex

%----------------------------------------------------------------------------------------
%	CONDITIONALS
%----------------------------------------------------------------------------------------				
\sectiontitle{Conditionals}

\command{\texttt{if (a == 4) \{...\}}}{Do if $a$ is exactly 4}
\command{\texttt{if (b < 60) \{...\}}}{Do if $b$ is less than 60}
\command{\texttt{if (c > -3) \{...\}}}{Do if $c$ is more than -3}

\textbf{Else:} Do something else when not true\\
\texttt{if (condition) \{...\} else \{...\}}\\

%----------------------------------------------------------------------------------------
%	IMAGES
%----------------------------------------------------------------------------------------

\sectiontitle{Images}

\command{\texttt{img = loadImage(path)}}{Assign to variable}
\command{\texttt{image(img, x, y)}}{Show image at $(x,y)$}
\command{\texttt{image(img, x, y, height, width)}}{Set size}
\command{\texttt{tint(r, g, b)}}{Color images}
\command{\texttt{noTint()}}{Undo image coloring}
\command{\texttt{imageMode(MODE)}}{Change display mode}
\command{\texttt{filter(FILTER)}}{Use image filter}

\textbf{Webcam}\\
\command{\texttt{webcam = createCapture(VIDEO)}}{Show cam}
\command{\texttt{webcam.hide()}}{Hide cam}

%----------------------------------------------------------------------------------------
%	SOUND
%----------------------------------------------------------------------------------------

\sectiontitle{Sound}

\command{\texttt{mp3 = loadSound(path)}}{Assign to variable}
\command{\texttt{mp3.play()}}{Play}
\command{\texttt{mp3.stop()}}{Stop}
\command{\texttt{mp3.setVolume(v)}}{$v$ ranges from 0 to 1}
\command{\texttt{mp3.pan(p)}}{$p$ ranges from -1 to 1. 0 is center.}


%----------------------------------------------------------------------------------------
%	FOOTNOTE
%----------------------------------------------------------------------------------------

\vspace{\baselineskip}
\linethickness{0.5mm} % Thickness of the footer line
{\color{mygray}\line(1,0){30}} % Print the line with a custom color

\footnotesize{
Created by Kristian Gårdhus Wichmann, 2020\\

Based on \url{www.latextemplates.com/template/cheatsheet}\\
				
Released under the MIT license.
}

%----------------------------------------------------------------------------------------

\end{minipage} % End the third column of text
} % End the third division of the page
\end{picture} % End the container for the entire page

%----------------------------------------------------------------------------------------

\end{document}